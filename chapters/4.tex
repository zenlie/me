\section{Pengertian Database}
\hfill \break
Database atau basis data adalah kumpulan berbagai data dan informasi yang tersimpan dan tersusun di dalam komputer secara sistematik yang dapat diperiksa, diolah atau dimanipulasi dengan menggunkan program komputer untuk mendapatkan informasi dari basis data tersebut. 
\hfill \break\
Istilah database sendiri mengacu pada koleksi data-data yang saling terkait satu sama lain dimana tujuan database dapat digunakan untuk mengelola data dengan lebih efektif dan efisien. 
\section{Fungsi Database}
\hfill \break\
Setelah memahami apa itu database, maka kita harus mengetahui apa itu fungsi database:
\begin{enumerate}
\item Mengelompokkan data dan informasi sehingga lebih mudah dimengerti
\item Mencegah terjadinya duplikat data maupun inkonsistensi data
\item Mempermudah proses penyimpanan,akses,pembaharuan, dan menghapus data.
\item Menjaga kualitas data dan informasi yang diakses sesuai dengan yang di-input.
\item Membantu proses penyimpanan data yang besar.
\item Membantu meningkatkan kinerja aplikasi yang membutuhkan penyimpanan data.
\end{enumerate}
\section{Manfaat Database}
Berikut beberapa manfaat dari menggunakan database yang bisa didapatkan jika bekerja dengan sistem database:
\begin{enumerate}
\item Tidak terjadinya redudansi Basis Data
\hfill \break
Database mampu meminimalkan terjadinya redudansi artinya redudansi sendiri itu merupakan terjadinya data-data ganda dalam berkas-berkas yang berbeda. 
\item Integritas Data Terjaga
\hfill \break
Database memastikan integritas data yang tinggi dimana database akan memastikan keakuratan,aksesbilitas, konsistensi dan juga kualitas tinggi pada suatu data.
\item Independensi Berbagai Data
\hfill \break
Database menjaga independensi data dimana orang lain tidak dapat merubah data meskipun data bisa diakses.
\item Kemudahan berbagai Data
\hfill \break
Menggunkan perangkat lunak database bisa digunakan untuk berbagi data atau informasi dengan sesa pengguna lainnya.
\item Menjaga Keamanan Data
Database menjamin keamanan suatu informasi data, dimana anda bisa meyisipkan kode akses untuk data-data tertentu yang tidak bisa diakses bersama.
\item Kemudahan Akses Data
\hfill \break
Dengan database bisa memudahkan untuk mengakses dan mendapatkan data karena semua data terorganisir dengan baik.
\end{enumerate}

\section{Maria DB}
\hfill \break
Disini kita akan mencoba untuk membahas berbagai macam tutorial mengenai Maria DB seperti cara install database MariaDB,Syntax,Tipe Data,koneksi,database,membuat database,memilih databsae,membuat tabel,operasi CRUD,cara insert,cara limit,cara update,cara delete,statement dan berbagai perintah bisa digunakan dalam MariaDB.
\hfill \break
Nah, sebelum masuk untuk mempelajari MariaDB ada baiknya jika teman-teman sudah mempelajari atau mengetahui dasar-dasar perintah MYSQL.
\hfill \break
MariaDB adalah proyek berbasis komunitas dari sistem manajemen basis data relasional MYSQL. MariaDB adalah teknologi database open source dan relasional yang dapat digunakan sebagai pengganti MYSQL.Maria DB dikembangkan oleh pengembang asli MYSQL yang khawatir setelah MYSQL diakuisisi oleh Oracle.
\hfill \break
Maria DB adalah relasional database manajemen sistem yang meyimpan data kedalam tabel-tabel yang ada didalam database.Primary Key dan Foreign Key digunakan untuk membangun relasi antar beberapa tabel yang berbeda.
\hfill \break
Relasional databsae manajemen sistem (RDBMS) memiliki beberapa fitur seperti berikut ini:
\begin{enumerate}
\item RDBMS memfasilitasi Anda untuk menerapkan sumber data dengan tabel, kolom, dan indeks.
\item RDBMS menyediakan integritas referensi antar baris dari beberapa tabel.
\item Hal ini digunakan untuk secara otomatis untuk memperbarui indeks.
\item RDBMS dapat diguanakan untuk menafsirkan query SQL dan operasi dalam memanipulasi atau sumber data dari tabel.
\end{enumerate}
\section{Istilah yang digunakan dalam RDBMS}
\hfill \break
Berikut ini adalah beberapa istilah yang digunakan dalam relasional database manajemen sistem pada MariaDB:
\begin{enumerate}
\item Datbase : Database merupakan suatu wadah yang berisi tabel-tabel yang berisi data
\item Table : Tabel merupakan struktur matrix yang berisi data.
\item Coloumn : Coloumn(kolom) adalah suatu elemen data. Coloumn merupakan suatu struktur yang menyimpan data dengan tipe yang sama.
\item Row : Row atau abris adalah struktur dimana suatu data disimpan, Row biasa disebut juga dengan tuple, entry atau record.
\item Primary Key : Primary Key merupakan suatu nilai yang unik. Nilai yang berupa Primary Key tidak dapat muncul dua kali didalam tabel yang sama.
\item Foreign Key : Foreign Key biasanya digunakan untuk menghubungkan dua buah tabel yang berbeda.
\end{enumerate}
