\section{Contoh studi kasus penggunaan aplikasi ini}
Smart Gudang merupakan sebuah aplikasi yang dirancang khusus untuk manajemen barang dimana cara kerja dari aplikasi ini dengan menunjukan scan barcode ke arah barang. Aplikasi ini ditujukkan bagi masyarakat yang memiliki usaha kecil yang terbatas biaya karena dilihat dari permasalahan yang ada dalam lingkup pergudangan bawah menengah masih banyak sekali kesulitan dalam hal pencatatan data sehingga tidak mengefisiensikan waktu. Dan dengan adanya aplikasi ini masyarakat yang memiliki usaha kecil tidak perlu merogoh banyak biaya unutk membeli alat scanner, cukup dengan menggunakan smartphone yang telah terinstall aplikasi scanner ini.

Tujuan
1.Mengefisiensikan waktu dengan mengubah teknologi tradisional menjadi teknologi modern
2.Mengganti device khusus scanner dengan android yang telah terintstall aplikasi smart Gudang
3.Mempermudah manajemen barang masuk dan barang keluar 

Manfaat
1.Mengefisiensi waktu dalam memanajemen barang masuk dan barang keluar 
2.Mengurangi biaya dalam pmenuhan alat device (alat scanner)
3.Masyarakat di permudah dalam memanajemen barang

\section{Manfaat Database}
Berikut beberapa manfaat dari menggunakan database yang bisa didapatkan jika bekerja dengan sistem database:
\begin{enumerate}
\item Tidak terjadinya redudansi Basis Data
\hfill \break
Database mampu meminimalkan terjadinya redudansi artinya redudansi sendiri itu merupakan terjadinya data-data ganda dalam berkas-berkas yang berbeda. 
\item Integritas Data Terjaga
\hfill \break
Database memastikan integritas data yang tinggi dimana database akan memastikan keakuratan,aksesbilitas, konsistensi dan juga kualitas tinggi pada suatu data.
\item Independensi Berbagai Data
\hfill \break
Database menjaga independensi data dimana orang lain tidak dapat merubah data meskipun data bisa diakses.
\item Kemudahan berbagai Data
\hfill \break
Menggunkan perangkat lunak database bisa digunakan untuk berbagi data atau informasi dengan sesa pengguna lainnya.
\item Menjaga Keamanan Data
Database menjamin keamanan suatu informasi data, dimana anda bisa meyisipkan kode akses untuk data-data tertentu yang tidak bisa diakses bersama.
\item Kemudahan Akses Data
\hfill \break
Dengan database bisa memudahkan untuk mengakses dan mendapatkan data karena semua data terorganisir dengan baik.
\end{enumerate}

