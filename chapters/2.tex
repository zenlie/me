\section{Sejarah PHP}
\subsection{PHP/FI : Personal Home Page/Forms Interpreter}
Sejarah PHP bermula pada tahun 1994 keika programmer kelahiran Denmark yang sekarang berdomisili di Canada, Rasmus Lerdorf membuat sebuah script(Kode Program) dengan bahasa Perl unutk web pribadinya. Salah satu kegunaan script ini adalah untuk menampilkan resume pribadi dan mencatat jumlah pengunjung ke sebuah website.
\hfill \break
Dengan alasan untuk meningkatkan performa, Rasmus Lerdorf kemudian membuat ulang kode program tersebut dadlam bahasa C. Ia juga mengembangkannya lebih lanjut sehngga memiliki script tersebut dan memiliki kemampuan unutk memproses form HTML dan berkomunikasi dengan database.
\hfill \break
Lerdorf menyebut kode program ini sebagai Personal Home Pages/Form Interpreter atau PHP/FI. Inilah asal mula penanaman PHP digunakan. PHP/FI dapat digunakan untuk membuat aplikasi web dinamis sederhana.
\hfill \break
Lerdorf kemudian merilis kode tersebut ke publik dengan sebutan Personal Home Page Tools9PHP Tools)version 1.0. 
\subsection{PHP/FI : Personal Home Page/Forms Interpreter 2}
\hfill \break
Seiring dengan pengembangan dan penambahan fitur web pada saat itu, April 1996, Rasmus Lerdrof mengumumkan PHP/FI Versi 2.0. PHP versi 1 sebenernya sudah mencukupi, namun performa yang dihasilkan dirasakan belum cukup, sehingga butuh penambahan fitur lanjutan.
\subsection{PHP:Hypertext Preprocessor 3}
\hfill \break
Evolusi PHP berikutnya terjadi pada pertengahan tahun 1997, PHP versi 2 telah menarik banyak perhatian programmer, namun bahasa ini memiliki masalah dengan kestabilan yang kurang bisa diandalkan. Hal ini di karenakan Lerdorf hanya bekerja sendiri untuk mengembangkan PHP.
\hfill \break
Dengan dukungan banyak programmer lainnnya, Proyek PHP secara perlahan beralih dari proyek satu orang menjadi proyek massal yang lebih di akrab kita kenal sebagai open-source project. PHP selanjutnya dikembangkan oleh The PHP Group yang merupakan kumpulan banyak programmer dari seluruh dunia.
\hfill \break
Perilisan PHP Versi 3 juga ditandai dengan perubahab singkatan PHP yang sebelumnya PHP/FI:Personal Home Pages Tools, menjadi PHP : Hypertext Preprocessor. Kepanjangan PHP sebagai PHP:Hypertext Preprocessor disebut juga sebagai kepanjangan rekursrif, sebuah istilah dalam pemrograman diaman suatu fungsi memanggil dirinya sendiri.
\hfill \break
Setelah perilisan PHP 3.0, PHP semakin populer digunakan di seluruh dunia. Dan sejak saat itu, penggunaan PHP sebagai bahasa pemrograman web menjadi sebuah standar bagi programmer.
\subsection{PHP:Hypertext Preprocessor 4}
\hfill \break
Segera setelah, Zeev Suraski,Andi Gutmans dan juga brbagai programmer di seluruh dunia mengembangkan PHP lebih jauh lagi dengan memperkenalkan banyak fitur lanjutan, seperti layer abstraksi antara PHP dengan web server, menambahkan mekanisme thread-safety dan two-stage parsing. Parsing baru ini dikemabangkan oleh Zeev dan Andi dan dinamakan Zend engine. Akhirnya pada 22 May 2000 diluncurkan PHP 4.0
\hfill \break
PHP versi 4 jga menyertakan fitur pemrograman objek/{object Oriented Programming, walaupun belum sempurna.
\subsection{PHP:Hypertext Preprocessor 5}
\hfill \break
Versi PHP terakhir hingga saat ini, yaitu PHP 5.x diluncurkan pada 13 juli 2004. PHP 5 telah mendukung penuh pemrograman object dan peningkatan performa melalui Zend engine versi 2.
\hfill \break
Beberapa penambahan fitur meliputi PDO(PHP Data Object) untuk pengaksesan database, closures, tarit dan namespaces.
\subsection{PHP:Hypertext Preprocessor 6}
\hfill \break
Versi lanjutan dari PHP , yakni PHP 6.x sebenernya telah lama dikembangkan, bahkan sejak tahu 2005. Fokus pengembangan PHP 6 terutama dalam mendukung Unicode agar PHP bisa mendukung berbagai jenis karakter bahasa non-latin.
\hfill \break
Namun karena beberapa alasan seperti kurangnya programmer dan performa yang tidak memuaskan, pengembangan PHP 6 dihentikan dan fitur yang ada dimasukkan ke dalam PHP 5. 
\subsection{PHP:Hypertext Preprocessor 7}
\hfill \break
Pada tanggal 3 Desember 2015, PHP 7 resmi dirilis. Perubahan yang paling terlihat adalah peningkatan performa. Menggunakan Zend Engine 3, PHP 7 di klaim berjalan 2 kali lebih cepat daripada PHP 5.6. Proyek ini menggunakan pendekatan modern agar PHP diproses dengan lebih cepat seperti memakai teknik just-in-time(JIT) compiler.
\hfill \break
Walaupun terkendala dengan perilisan PHP versi 6. PHP 7 saat ini menjadi versi PHP terbaru dan versi yang disarankan.
